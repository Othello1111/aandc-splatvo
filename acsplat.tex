\documentclass[final,authoryear,5p,times,twocolumn]{elsarticle}
%\documentclass[authoryear,preprint,review,12pt]{elsarticle}

%\usepackage{epstopdf}
\usepackage{graphicx}
\usepackage{amssymb}
\usepackage[pdftex,pdfpagemode={UseOutlines},bookmarks,bookmarksopen,colorlinks,linkcolor={blue},citecolor={green},urlcolor={red}]{hyperref}
\usepackage{hypernat}
%\usepackage{url}
\usepackage{upquote}
\usepackage{upgreek}

\journal{Astronomy \& Computing}

%% Local commands

\newcommand{\qjras}{QJRAS}
\newcommand{\apj}{ApJ}
\newcommand{\aap}{A\&A}
\newcommand{\aspconf}{ASP Conf.\ Ser.}

%% Protocols

\newcommand{\datalink}{\texttt{DataLink}}
\newcommand{\ssap}{\texttt{SSAP}}
\newcommand{\tsap}{\texttt{TSAP}}
\newcommand{\regtap}{\texttt{RegTap}}
\newcommand{\obstap}{\texttt{ObsTap}}
\newcommand{\tap}{\texttt{TAP}}
\newcommand{\obscore}{\texttt{ObsCore}}
\newcommand{\getdata}{\texttt{getData}}
\newcommand{\votable}{VOTable}
\newcommand{\plastic}{\texttt{PLASTIC}}
\newcommand{\samp}{\texttt{SAMP}}
\newcommand{\adql}{ADQL}

%% Standards
\newcommand{\vounits}{VOUnits}

%% Software applications
\newcommand{\splat}{\textsc{splat}}
\newcommand{\splatvo}{\textsc{splat-vo}}
\newcommand{\topcat}{\textsc{topcat}}
\newcommand{\vospec}{\textsc{VOSpec}}
\newcommand{\specview}{\textsc{Specview}}
\newcommand{\dachs}{\textsc{DaCHS}}
\newcommand{\aladin}{\textsc{Aladin}}
\newcommand{\gaia}{\textsc{gaia}}
\newcommand{\KAPPA}{\textsc{kappa}}
\newcommand{\specx}{\textsc{specx}}
\newcommand{\spefo}{\textsc{spefo}}
\newcommand{\intep}{\textsc{intep}}
\newcommand{\pleinpot}{\textsc{pleinpot}}

%% Links
\newcommand{\ascl}[1]{\href{http://www.ascl.net/#1}{ascl:#1}}


\begin{document}
\begin{frontmatter}

\title{Spectroscopic Analysis in the Virtual Observatory Environment with SPLAT-VO}

\author[OND]{Petr \v{S}koda\corref{cor1}}
\ead{skoda@sunstel.asu.cas.cz}
\author[DUR]{Peter W. Draper}
\author[HDB]{Margarida Castro Neves}
\author[VSB]{David Andre\v{s}i\v{c}}
\author[COR]{Tim Jenness}
\cortext[cor1]{Corresponding author}

\address[OND]{Astronomical Institute of the Academy of Sciences,Fri\v{c}ova~298, 251\,65, Ond\v{r}ejov, Czech Republic}
\address[DUR]{Department of Physics, Institute for Computational
Cosmology, University of Durham, South Road, Durham DH1 3LE, UK}
\address[HDB]{Universit\"a{}t Heidelberg, Astronomisches Rechen-Institut,
M\"o{}nchhofstra\ss{}e 12--14, 69120 Heidelberg, Germany}
\address[VSB]{Department of Computer Science, Faculty of Electrical
Engineering and Computer Science, V\v{S}B --- Technical University of Ostrava\\
 17. listopadu 15, 708 33 Ostrava-Poruba, Czech Republic}
\address[COR]{Department of Astronomy, Cornell University, Ithaca, NY 14853, USA}


\begin{abstract}
  \splatvo\ is a powerful graphical tool for displaying, comparing, modifying
  and analysing astronomical spectra, as well as searching and retrieving
  spectra from services around the world using Virtual Observatory (VO)
  protocols and services. The development of \splatvo\ started in 1999, as part
  of the Starlink StarJava initiative, sometime before that of the VO, so
  initial support for the VO was necessarily added once VO standards and
  services became available. Further developments were supported by
  the Joint Astronomy Centre,
  Hawaii until 2009. Since end of 2011 development of \splatvo\ has been
  continued by the German Astrophysical Virtual Observatory, and the
  Astronomical Institute of the Academy of Sciences of the Czech Republic.
  From this time several new features have been added, including support for
  the latest VO protocols, along with new visualisation and spectra storing
  capabilities. This paper presents the history of \splatvo, it's
  capabilities, recent additions and future plans, as well as a discussion on
  the motivations and lessons learned up to now.
\end{abstract}
\begin{keyword}
Spectral Analysis \sep Virtual Observatory \sep VO \sep SPLAT-VO \sep StarJava
\end{keyword}
\end{frontmatter}

%-------------------------------------------------------
\section{Introduction}

There are many tools for the analysis of astronomical spectra.  The most
commonly used of these are the various packages of MIDAS
\citep[][\ascl{1302.017}]{1992ASPC...25..115W}, IRAF
\citep[][\ascl{9911.002}]{2012ASPC..461..595F}, and Starlink
\citep[][\ascl{1110.012}]{1982QJRAS..23..485D}, together with scripts
written in python \citep[e.g.,][]{2013A&A...558A..33A} and IDL \citep[e.g.,][]{1993ASPC...52..246L}.
Very advanced processing is still often done in custom
(especially FORTRAN) programs used by small communities.  These tools tend to
work with specially formatted files (quite often ASCII tables) stored on local
disks.

The common tasks for an astronomer during spectral analysis consists of finding
the required spectra in various archives, downloading them, understanding and
extracting metadata (such as date of exposure, spectral range, and flux units),
converting to the custom format used by the particular tool, visualizing
selected spectra and  removing the bad ones (for example if they are
noisy or have excessive numbers of spikes). They can finally
run some massive processing or analysis on an
improved list.  Very often the measurement of some parameter, such as the radial
velocity or equivalent widths of selected lines, is accomplished interactively
and the spectra must be adjusted by zooming, panning and scaling.

A real scientific analysis of thousands of spectra is a very
tedious task even from the same instrument, but the real modern challenge is an
analysis of multi-wavelength spectra from different instruments, where all the
metadata are different, there are various flux and wavelength, frequency or
energy units  and where the formats of storage also vary (including simple FITS, binary
tables, and ASCII lists).

Making the handling of spectra obtained from world-wide distributed
heterogeneous archives simple and efficient was the main motivation for
developing the International Virtual Observatory Alliance Simple Spectral
Access Protocol \citep[IVOA \ssap;][]{ssap}. The key elements of this
\texttt{http}-based protocol are a standard spectral server discovery mechanism (a
Google for astronomical data) called IVOA Registry \citep{registry} and  the
standard \votable\ data format\citep{2004tivo.conf..118O},
that is used to describe the metadata of discovered spectra in a limited vocabulary of
obligatory parameters.

\subsection{The SSA Protocol}
%
\ssap\ allows astronomers to find all the world-wide VO-compatible archival
spectra of their favourite objects, and it can also be used to get spectra for all
the objects in a `circular' region of interest on the sky. These simple
queries can be refined using spectral ranges, dates of observation as well as
more subtle criteria like spectral resolution or target class (even
spectral type) if the server supports such refinement. More capable servers
provide extended facilities for converting spectra on-the-fly to the required
format, and can create previews of spectra in \texttt{PNG} or deliver
the fluxes and wavelengths in ASCII or CSV tables.

\subsection{SSAP-compatible  Visualization Clients}

Although processing large amounts of data is more convenient when
using \ssap\ from a scripting language for batch processing, a crucial
part of understanding data is only gained by using advanced visualization and interactive
exploration.  Therefore, a very important part of the VO
infrastructure is represented by VO-compatible spectra visualization
clients. There are currently three major tools for interactively manipulating
spectra from the VO.

\vospec\ \citep[][\ascl{1205.011}]{2005ASPC..347..198O}, developed by ESA, is
designed strictly on principles of IVOA standards and was rather focused on
the analysis of high energy and space-based spectra with absolute flux
calibration, lacking support for some common functions and custom data formats
used in ground-based stellar spectroscopy. The most critical 1D FITS image
format was added just recently. \vospec\ contains quite complex non-linear
model-fitting package and can handle photometric points as well as Spectral
Energy Distributions (SEDs).

The other two clients were developed as general spectral analysis
tools prior to the conception of the VO and have been extended to add
support for querying and downloading spectra using \ssap.  The
advantage of this approach is the support for many legacy data formats
and various domain-specific functionality, better tuned to
requirements of scientists not familiar with VO technology.

\specview\ \citep[][\ascl{1210.016}]{2002SPIE.4847..410B}, developed at
STScI, was intended mainly for handling spectra from Hubble Space
Telescope instrumentation and some other NASA missions, but gradually
transformed into a VO-compatible general client supplemented with a
library of theoretical spectra and various models for SED generation.
While \specview\ is a very powerful tool compatible with a lot of very
special formats from \emph{HST}, \emph{IUE}, \emph{FUSE}, \emph{ISO}
and \emph{GALEX} satellites, allowing for the advanced comparison of
observations with a wide range of theoretical spectral models (such as
power laws, Bremsstrahlung, emission line profiles, interstellar
extinction, dusty rings or whole Kurucz library of spectral types), it
is less flexible in the support of various \ssap\ features and is falling
behind the rapid evolution in IVOA standards.

In contrast with \specview, the last VO-compatible client for spectra analysis,
\splatvo\ \citep[][\ascl{1402.008}]{sun243} was revived recently and currently
is under intensive development serving as a test-bed for newly proposed VO
standards and special features.


%\subsection{VO access in practical examples}

%-------------------------------------------------------

\section{History of SPLAT-VO}
\subsection{Genesis}

The \splatvo project is an extension to an older one simply called
\splat\ \citep[][\ascl{1402.007}]{2002ASPC..281..513B}.  The name \splat\
is based on the words SPectraL Analysis Tool, an application for the
display and analysis of spectral data. The development of \splat\ itself
was started by the Starlink Project \citep{1982QJRAS..23..485D} in
1999 in response to a perceived need within the UK astronomy community
for a spectral domain tool with capabilities similar to those of the
successful \gaia\ tool \citep[][\ascl{1403.024}]{2000ASPC..216..615D}
that supported imaging data. \splat\ was also a leader project for
moving future developments into the Java language, with the obvious
benefits of improved portability, modern language capabilities (OOD,
OOP) and core-level support for features like UIs and internet
protocols and services. This work eventually led to the formation of
what became the StarJava project, now hosted at
\url{https:://github/Starlink/starjava}, where the source code for
\splatvo\ and other StarJava projects, like \topcat\
\citep[][\ascl{1101.010}]{2005ASPC..347...29T}, can be freely
downloaded.

\subsection{SPLAT-VO}

The first VO capabilities added to \splat\ where released in 2005
\citep{2005ASPC..347...22D}, as support for querying VO registries for
Simple Spectral Access Protocol servers was added. A selection of
these servers could then be queried for any spectra they contained
within a region on the sky and these could then be selected and
displayed. \splat\ already had the ability to match a wide range of
spectral coordinate systems and flux units, so the spectra could be
displayed within the same plots for direct comparison. Since then \ssap\
support has been developed further with version 1 of the standard and
various enhancements like being able to query for an object, not just
a region.

The second phase of VO developments in \splatvo\ were the incorporation
into the VO-desktop, so that spectra and tables could be shared
between applications.  Initially, in 2006, this used the \plastic\
\citep{2007ASPC..376..511T} protocol and then, more recently, \samp\
\citep{2012ASPC..461..279T}.


{\color{red}
The institutions for all these people are already listed at the
top of the paper so it's not obvious to me why we need to mention
them. We don't say \emph{Peter Draper of Durham University was the lead
programmer} so my inclination would be to not name names or
institutions and just state GAVO has taken a lead in development. The
rest is explicit in the author list.
}

In 2011, the VO aspects of \splatvo\ have been taken over by Margarida
Castro Neves, from the German Astrophysical Virtual Observatory (GAVO) in
Heidelberg, Germany.  Due to the need for a spectral domain application with
VO capabilities and good scientific analysis functions, GAVO  decided to resume
development once the original funding had stopped. In this way \splatvo\ can
continue to be improved, and kept up-to-date with the changing requirements of
both end users and data providers.  This development  is done in cooperation
with Petr \v{S}koda (Astronomical Institute of the Academy of Sciences of the
Czech Republic), who contributes by defining the user requirements and
suggesting modifications of scientific functions (with emphasis put on optical
stellar spectroscopy of  emission line stars).  David Andre\v{s}i\v{c},
from Technical University, Ostrava, has been implementing new analytic
functionality (see section.~\ref{davids_functions}) since 2012.

\section{SPLAT-VO Internals}

\subsection{Data Formats}

The \splatvo\ spectral data model is specified by a single internal interface
that is concretely implemented for the various data formats that are
supported. Having a single interface decouples the data model and makes it
practical to support a wide range of formats, but necessarily means that some
format specific features will not be supported and can be lost when making
transformative changes like saving back to file.

The \splatvo\ data model is simple only requiring that spectra have a name,
coordinate system and some data values. Other supported items are data errors,
fuller descriptions, keyed properties (key, value, description), data units
and labels, and the underlying dimensionality (if the spectrum is being
extracted from a 2D or 3D array). Spectra can also have mutable properties,
like the names of the table columns that contain the various data values.

Current implementations give \splatvo\ access to spectra in simple text
files, FITS \citep{2010A&A...524A..42P}, NDF \citep{ndfjenness}, NDX
\citep{2003ASPC..295..221G}, as well as FITS and VO tables. Tables are
supported using the STILTS
\citep[][\ascl{1105.001}]{2006ASPC..351..666T} library. In addition to
these spectra can also be memory based so that they can become
modifiable and be derived from data with higher dimensionalities.

\subsection{Internal spectral handling}

The presentation of all data types through a single interface works especially
well when coupled with a further, higher-level, abstraction class. This
abstraction is in effect how all the analysis and visualisation code
throughout \splatvo\ see a `spectrum' (these two elements that decouple an
abstraction from the underlying implementations are in fact the well
established `Bridge Pattern').

This use of a single abstraction to all data formats makes it easier to write
consistent code and extend the basic internal spectrum with new facilities
(methods that do things like extraction sections, replace BAD values etc.) as
well as associate useful properties like graphics and formatting rendering
hints, data and coordinate range limits. Extending the underlying model is
also easy as new features strictly need only be handled in the abstraction
(although clearly that would not be interesting) and new spectral types can
extend the base abstraction (the basic spectral type is extended to also offer
spectra that render as line identifiers and cached copies of remote spectra).


\subsection{Presentation}

The presentation of spectra is based on the model-view-controller pattern that
pervades Java UIs and the same spectral data can be seen and operated on in
many different guises, that is as a number of line plots, tables, property
lists etc. Memory based spectra are also directly modifiable (as simple edits
to table cells or generally using expressions). Through the
model-view-controller infrastructure all changes are propagated to all views.


\subsection{Coordinate systems and units}

Early in the design of \splat\ it was decided that it would not be possible to
replicate the considerable effort that had already been put into a
comprehensive system for handling coordinate systems and units, including full
support for reading and writing FITS WCS \citep{2006A&A...446..747G}. This led to the development of a JNI
library for the Starlink AST library \citep[][\ascl{1404.016}]{1998ASPC..145...41W,2012ASPC..461..825B}, JNIAST. AST
itself is written in standard C so compiling it on many platforms is
relatively easy, but, as discussed in Sec.\ \ref{sec:jniast-lesson}, this remains a portability issue at the heart of \splat\
limiting one of the initial aims.

Some of the advanced features that \splatvo\ offers based on AST are support
for spectral coordinates in wavelength, velocity, frequency and energy. These
can have standards of rest, so transformations of rest frame are also
supported (important for velocity and laboratory data). \splatvo\ also supports
double side-band spectra for sub-millimeter spectra and flux unit matching. Many of these features
have been improved in AST over time and \splatvo\ gains from this.

AST supports all the units specified in the FITS standard
\citep{2010A&A...524A..42P} and can calculate scaling factors to allow
matching of spectra with different flux units. Adding full \vounits\
support \citep{vounits} to \splatvo\ would currently require that AST
is updated to match the standard. This would have the added benefit of
expanding support for \vounits\ to the entire Starlink software
collection and in particular enhance the unit support of the VO
aspects of \gaia\ \citep{2009ASPC..411..575D}.


\subsection{Scripting of SPLAT-VO}

The various classes of \splatvo\ may be called directly from the BeanShell
scripting language \citep{niemeyer2013learning}, in principle this allows
complex workflows to be created, and is used to create example scripts that
are part of the standard distribution. These support the batch fitting of
single Gaussian line profiles, as well as composite models consisting of any
number of Gaussian, Lorentzian and Voigt profiles for deblending complex lines
(a feature that has never made it into the full UI). BeanShell scripts are
also used for controlling \splatvo\ from command line.

%-------------------------------------------------------

\subsection{The Power of SAMP in SPLAT-VO}

Although all \ssap\ clients have powerful capabilities, they are naturally
limited to their built-in features, which will not suit all the requirements
of a wide spectroscopic community. However, almost all VO-compatible tools,
including the important table processor \topcat\ and stellar atlas \aladin\
\citep[][\ascl{1112.019}]{2005ASPC..347..193O}, have built-in support for the Simple Application Message
Protocol (\samp), successor of \plastic. This supports the bringing together of
basic VO applications, like bricks, to build complex visualization and
processing pipelines allowing very complicated analyses of truly Big Data.
The astronomer may, for example, visualize the spectra of objects they click
on in some imaging survey frame displayed in \aladin, or identify the visual
appearance of galaxies from the spectra which have been visually selected in
\splatvo. \samp\ has successfully been used to send messages from a web browser
displaying previews of a number of spectra to \splatvo\ where they were
subsequently automatically downloaded and then analysed.

\section{SPLAT-VO Capabilities  for Spectra Analysis}

\begin{figure*}[t]
\begin{center}
\includegraphics[width=0.8\textwidth]{phiper-heros-stack.pdf}
\caption{The time evolution of the H$_\alpha$ line emission profile in
$\phi$~Per observations obtained with the HEROS spectrograph at Ond\v{r}ejov 2m
telescope in years 2001--2003. The stacking offset of 0.6  makes sure that the changes can be seen easily}

\label{fig:phiper-heros-stack}
\end{center}
\end{figure*}

\begin{figure*}[t]
\begin{center}
\includegraphics[width=0.8\textwidth]{zetoph2sp-id.pdf}
\caption{Spectrum of $\zeta$~Oph in H$_\alpha$ region  obtained with Ond\v{r}ejov Coud\`e
  spectrograph 700mm camera (green) and HEROS spectrograph (blue). The line
  identifications from custom line list of telluric lines is overplotted in violet. Note the deeper
  telluric lines from HEROS due to its almost double spectral resolving power.}
\label{fig:zetoph2sp-id}
\end{center}
\end{figure*}

\splatvo\ as a general purpose spectral analysis tool supports a very wide
range of capabilities required by astronomers for every-day work outside of
the VO. Input spectra may be loaded directly from local disk files in a number
of formats, including several types of multicolumn ASCII tables,
multi-extension FITS with binary tables or 1D spectrum FITS image, as well as
Starlink NDF.

The list of capabilites of \splatvo\ is enormous and are fully described in
the associated documentation \citep[SUN/243;][]{sun243} which is also
available as online help and includes example data from
multi-wavelength observations.

Here we give a short overview of features not relevant to VO protocols.
Several features are very specific to the analysis of hot emission
line stars.

\begin{itemize}

\item Visual stacking of spectra introducing either constant vertical offsets
  or offsets based on functions of values of certain FITS keywords. For example
  the line profiles may be ordered by the circular phase computed for a
  certain period. Stacking is also useful for visualization of evolution of
  stellar line profiles in variable stars.
  Fig.~\ref{fig:phiper-heros-stack} shows the evolution of an emission line
  profile for the star $\phi$~Per observed with the HEROS spectrograph
  \citep{2002PAICz..90....1S} at
  the Ond\v{r}ejov 2m telescope during the years 2001--2003.

\item Measurement of radial velocities by the visual matching of a line
  profile with its mirrored image based on idea of oscilloscopic comparators.
  This is important for measuring asymmetric and distorted line profiles
  typical for hot emission stars. For a detailed explanation see
  \citet{2007IAUS..240..486P}. This feature can be operated with a batch-like
  mode to quickly assess a number of lines and or spectra using a \emph{visitor list}.

\item Continuum estimation using flexible curves similar to those expected to
  be drawn by a human. The \intep\ procedure based on Hermite polynomials
  \citep{1982PDAO...16...67H} has been successfully used for decades in
  fitting stellar continua (typically of emission-line stars such as Be stars and
  symbiotic novae) in the program \spefo\ \citep{1996ASPC..101..187S}. An
  overview of the advantages of this procedure is given in
  \citet{2008asvo.proc...97S}. Other curves types are also available such as
  \citet{Akima:1970:NMI:321607.321609}.

\item Powerful transformation of the spectral axis with respect to various
  coordinate systems; air and vacuum wavelength, optical and radio velocities,
  frequency, redshift, energy. The transformations can also include various
  standards of rest; topocentric, heliocentric, dynamic and kinematic local.

\item Built-in spreadsheet processor supported by special astronomical
  functions. Allows the modification/transformation of coordinates and data
  values. Astronomical functions include the spectral profile models.

\item Filtering with smoothing (mean, median, rebin) and denoising with wavelets.

\item Line fitting using Gaussian, Voigt or Lorentz profiles.

\item Statistics of selected regions, mean, median, sum, mode, variance, skew,
      kurtosis, rms, quantiles.

\item Region removal, replacement and extraction. Also replacement of part of
 spectra with an interpolated curve.

\item Comprehensive overlay graphics toolkit with annotations and many
  configurable options for the presentation of data. Export of publication quality
  output into \texttt{PNG}, \texttt{GIF}, and \texttt{EPS} formats.

\item Animation of spectra - convenient for showing line evolution or
  non-radial pulsations.

\item Identification of spectral lines using simple supplied ASCII line lists
  and built-in sets of common atomic and molecular lines (includes molecules in
  sub-millimeter region). Fig.~\ref{fig:zetoph2sp-id} shows the identification
  of telluric lines on a combination of spectra of $\zeta$~Oph obtained with
  two different spectrographs of the Ond\v{r}ejov 2m telescope.

\item Simple operations on spectra such as addition, subtraction, division and
  continuum fitting.

\end{itemize}


\section{SPLAT-VO SSAP interface}

The \ssap\ interface in \splatvo\ is clearly the most important tool for working
with the VO and remains under active development. The task of the interface is
to help the user construct a valid \ssap\ query by filling out a simple form of
parameters, and then arrange for that query to to sent to a list of selected
\ssap\ servers and finally make the query response available so that any spectra
can be selected for download.

Queries can involve supplying the coordinates (\texttt{POS}) and size
(\texttt{SIZE}) defining a 'circular' region on the sky (a so called `cone
search'), or just the name of an object and size, the name being resolved into
a position by the CDS name resolver Sesame. Other form elements provide
refinement of the query to include things like spectral ranges
(\texttt{BAND}), the date and time of observation (\texttt{TIME}), status of
the wavelength calibration (\texttt{WAVECALIB}), flux calibation
(\texttt{FLUXCALIB}) and data format (\texttt{FORMAT}).  In fact recent
additions to \splatvo\ now offer control of all the optional parameters
supported by a selected \ssap\ server.

Once spectra have been selected they are downloaded from the \ssap\ server
(using the \texttt{URL} \texttt{accref} returned in the \ssap\ response) they
can be worked on in the same fashion as any locally available spectra and
saved to local disk etc.

The list of \ssap\ servers may be updated using one of several VO registries.
Local or development services not yet registered may be added manually. The
server list may be edited so that the query is only sent to collections of
servers with say similar data or can be restricted to only one service that
may be queried for detailed specific operations.

\begin{figure*}[Ht]
\begin{center}
\includegraphics[width=0.8\textwidth]{hst_query.pdf}
\caption{The \ssap\ query for absolutely flux calibrated spectra of $\phi$~Per
  with results obtained from the \emph{HST} GHRS service}
\label{fig:hst_query}
\end{center}
\end{figure*}


\begin{figure*}[t]
\begin{center}
\includegraphics[width=0.8\textwidth]{iuehst2.pdf}
\caption{Composite plot of UV spectra of $\phi$~Per from the \emph{HST} GHRS
  spectrograph (thick blue short pieces) selected in the query shown in
  Fig.~\ref{fig:hst_query} together with \emph{IUE} low resolution spectra
  that were selected earlier (thin lines: green SWP, pink LWR camera)}
\label{fig:iuehst2}
\end{center}
\end{figure*}

As an example, Fig.~\ref{fig:hst_query} shows the search of absolutely
calibrated spectra of the emission star $\phi$~Per. We have selected the
Hubble Space Telescope with the Goddard High Resolution Spectrograph (at the
opened tab, note also the unfolded service description on the left side) and
the \emph{IUE} low resolution spectra (selected in the other tab).  The
composite plot of UV spectra from \emph{HST} GHRS and \emph{IUE} low
resolution spectra is in Fig.~\ref{fig:iuehst2}.

\subsection{ Theoretical Spectra Access}

In a similar way to observational spectra, synthetic spectra can also be
obtained from libraries available in the VO (e.g., Kurucz models, Rauch's
non-LTE model spectra, TLUSTY hot stars, Salpeter, Dusty). They are selected
by a radio button, which changes the role of parameters used in the query. The
extended protocol called Theory-\ssap\ \citep[hereafter TSAP;][]{ssap} uses
various physical parameters like $T_{\rm eff}$, $\log g$, metalicity etc. for
queries instead of position, wavelength, region and date/time range.
Unfortunately obligatory metadata do not exist for all required physical
parameters, so it is necessary to just try out what is available on a given
server.  It is then necessary to individually select suitable spectra from a
multidimensional table combining all the parameter ranges required.
Fig.~\ref{fig:TSAP-query} and Fig.~\ref{fig:TSAP-plot} show the query and
result of selecting several Kurucz models for Vega.


\begin{figure*}[t]
\begin{center}
\includegraphics[width=0.8\textwidth]{TSSA-query.pdf}
\caption{\tsap\ Query for Vega-like Kurucz models. Note the additional parameters.}
\label{fig:TSAP-query}
\end{center}
\end{figure*}


\begin{figure*}[t]
\begin{center}
\includegraphics[width=0.8\textwidth]{TSSA-plot.pdf}
\caption{Zoomed plot of Vega-like Kuruz models manually selected in the previous
\tsap\ Query window (see Fig.~\ref{fig:TSAP-query}). }
\label{fig:TSAP-plot}
\end{center}
\end{figure*}


%-------------------------------------------------------

\section{Recent Additions}


\subsection{Motivation}

As the VO evolves, adding new protocols and data models, \splatvo\ needs to
evolve accordingly.  \ssap, as the name says, is a simple protocol, and has
some limitations. Some features like cut-outs or flux calibration, which are
often needed, have to be performed in a second work step after downloading the
spectra. It is much more efficient to have it done on-the-fly at the server
and  to download the spectra exactly as needed. To overcome the limitations of
\ssap, the new \datalink\ protocol \citep{datalink} was used. \splatvo\
is one of the first client implementations of \datalink, in this way
also contributing to its testing and development from a client point of view.

Besides \ssap, spectra can also be retrieved using the \obscore\ data model
through Table Access Protocol  \citep[known as \obstap;][]{obstap}. Its implementation in
\splatvo\ allows spectra to be retrieved also from \obscore\ services, and
the \obstap\ \adql\ \citep{adql} search offers a more powerful search
mechanism as well. The implementation of new VO protocols like \obstap\
and \datalink\ have been done side-by-side with their server-side
implementation in \dachs\ \citep[Data Center Helper Suite;][]{dachs}.

\splatvo's user interface has been improved. An improved server
selection interface and \ssap\ search by metadata parameters are some of
the new features added, which are listed in the next subsections.

\subsection{New Features in VO Access}

\subsubsection{SSAP service selection}

In the earlier versions of \splatvo\ a list of services was presented, and
the user could select the services to be queried, as well as remove
uninteresting services from the list. Information about the services
of interest had to be gotten from somewhere else, as the service
metadata was not used.  This simple server selection interface has
been improved by a more detailed selection, users can choose a set of
services that may contain data of interest, based on the services
metadata taken from the registry (data source, waveband).

There are two ways of selecting services. The first way is to choose
according to data source (such as a survey, pointed observation, or simulation)
or waveband.  For example, the user wants to select services
containing pointed observation sources in optical wavelength, so this
can be selected in the interface and only the services which contain
this information in the metadata will be selected.  The other way is
useful in the case when users know exactly the services they want to
query, so they may create an own tag containing the chosen services,
and give it a name, such as ``My Favourite Services'', which will be saved
and loaded the next time \splatvo\ starts.

This implementation relies on the metadata information from the
registry services, which should be correct and complete. Unfortunately
many services information from the registries are not complete, or not
correct, what we hope will be fixed in the future.



\subsubsection{Manual addition of SSAP services}

\splatvo\ gets its list of \ssap\ services by querying a registry
service.  A new function has been added to allow inclusion of a new
service that is not (yet) registered. This is useful to access a
local, not public service, or to test a service before it's included
in the registry.

\subsubsection{Metadata parameters query}

In early versions, \splatvo\ contained just a simple cone search using
the \ssap\ obligatory parameters (RA, DEC, band, time, data format) and later
added wavelegth and flux calibration state, although much other metadata is
available on servers.  Now \splatvo\ also retrieves the metadata parameters for
each server, and a new user interface allows users to chose metadata
parameters and their values to perform more detailed queries.  This is
especially interesting when querying theoretical spectra by \tsap, which contain many
additional query parameters. Figure~\ref{fig:TSAP-query} shows the
use of additional metadata parameters.

\subsubsection{Simple HTTP authentication}

Simple HTTP authentication can be used to restrict access to data when it is published to the VO before it should be
publicly available, so the owner and a limited group of users can
access it before disclosure. The authentication can be done on a
service basis, where the whole service needs authentication to be
queried, or on a spectrum basis, where only some of the spectra from a
certain service require authentication to be downloaded. In any case,
an username/password window will appear when needed.

\subsubsection{GetData and DataLink}

To overcome \ssap\ limitations and allow server-side processing of
spectra like cut-outs, or format conversions, a protocol called
\getdata\ \citep{getData}  has been developed. After successful implementation and
testing on SPLAT-VO, the work on \getdata\ has been stopped. It has been
decided to implement these features using the newer \datalink\
protocol, which will probably be made available on several services in
the near future, and can also be used in different cases.  \datalink\
resources can provide links with several different ways to access the data,
covering \getdata\ functionality and more.
When parsing the service response from an \ssap\ query, which is in form
of a \votable, \splatvo\ looks for a \texttt{RESOURCE} element of type \texttt{service}.
An example is shown below:

{\tiny
\begin{minipage}{\textwidth}
\begin{verbatim}
<RESOURCE ID="aeoadowdpudn" type="service">
  <GROUP name="input">
    <PARAM arraysize="*" datatype="char" name="ID"
    ref="ssa_pubDID" ucd="meta.id;meta.main" value="">
      <DESCRIPTION> The publisher DID of the dataset of interest
      </DESCRIPTION>
    </PARAM>
    <PARAM arraysize="*" datatype="char" name="FLUXCALIB"
    ucd="phot.calib" utype="ssa:Char.FluxAxis.Calibration" value="">
      <DESCRIPTION>Recalibrate the spectrum.  Right now, the only
      recalibration supported is max(flux)=1  ('RELATIVE').</DESCRIPTION>
      <VALUES>
        <OPTION name="RELATIVE" value="RELATIVE"/>
        <OPTION name="UNCALIBRATED" value="UNCALIBRATED"/>
      </VALUES>
    </PARAM>
    <PARAM ID="aemoogtm" datatype="float" name="LAMBDA_MIN"
    ucd="par.min;em.wl" unit="m" value="">
      <DESCRIPTION>Spectral cutout interval, lower limit</DESCRIPTION>
      <VALUES>
        <MIN value="3.3696e-07"></MIN>
        <MAX value="8.7665e-07"></MAX>
      </VALUES>
    </PARAM>
    <PARAM ID="appadowdpudn" datatype="float" name="LAMBDA_MAX"
    ucd="par.max;em.wl" unit="m" value="">
      <DESCRIPTION>Spectral cutout interval, upper limit</DESCRIPTION>
      <VALUES>
        <MIN value="3.3696e-07"></MIN>
        <MAX value="8.7665e-07"></MAX>
      </VALUES>
    </PARAM>
  </GROUP>
  <PARAM arraysize="*" datatype="char" name="accessURL" ucd="meta.ref.url"
  value="http://dc.zah.uni-heidelberg.de/flashheros/q/sdl/dlget"/>
</RESOURCE>
\end{verbatim}

\end{minipage}
}

The \texttt{GROUP} element with name \texttt{input} contains the input
parameters that will be returned to the server.  It must contain an \texttt{ID}
parameter, which refers to the metadata that identifying the required
spectra (normally the \texttt{pubDID} parameter).  The other parameters in
this group are the ones for which the users can set values that will
be sent to the service as a request. The \texttt{accessURL} parameter defines
the service URL to which this request has to be sent.

In the \ssap\ response to a query, the services supporting \datalink\ will
be marked with the  ``\ding{34}''  icon. When one of these services is
selected, the user can activate the \datalink\ feature by clicking on
the button with same name. When activated, a window with a form to
enter values to the \datalink\ parameters will appear. While this window
is activated, every spectrum selected and downloaded from this service
will be processed according to the chosen parameters. When
de-selected, the spectra will be downloaded as they are in the \ssap\
service, without processing.

In the case described above, the \datalink\ resource is in the query
response from the \ssap\ service. In another case, after a query, some
services return a list of spectra pointing not to the spectral data to
be retrieved, but to a \votable\ containing only \datalink\ information
with the links to the URLs of the data. This \datalink\ resource has
then to be parsed by \splatvo\ in order to retrieve the selected spectrum.

\subsubsection{ObsTAP}

In addition to \ssap, spectra can be retrieved using \obstap. This service uses
the Table Access Protocol (\tap) to query metadata from the Observation
Data Model Core Components (\obscore). Currently there are few services
implementing it, and the \splatvo\ implementation is ongoing.  \obscore\
provide standard metadata attributes that can be used, through the
\adql\ query language, to perform more extensive and detailed queries
than in the case of \ssap\ data discovery.

In the current \splatvo\ implementation, users can select \obscore\ in the
main \splatvo\ window. The \obscore\ browser window will appear, which is in
part similar to the \ssap\ browser window. The user can chose either a
similar cone search interface like the one in \ssap, or a
simple \adql\ interface where some parameters can be set. For more
advanced queries, the user can directly edit the \adql\
expression. After sending the query the results tables will be
displayed, which are in functionality similar to the \ssap\ browser
window.

\subsubsection{Visualizing Light Curves using SSAP}

\begin{figure*}[t]
\begin{center}
\includegraphics[width=0.8\textwidth]{OGLE-SC7-127550_query.pdf}
\caption{Example of using \ssap\ for obtain multicolour light curves from OSPS}
\label{fig:OGLE-SC7-127550_query}
\end{center}
\end{figure*}

\begin{figure*}[t]
\begin{center}
\includegraphics[width=0.8\textwidth]{OGLE-SC7-127550_plot.pdf}
\caption{Multicolour light curves of Cepheid OGLE-SC7-127550 from
  OSPS. The manual modification of default line types into various
  markers and flip of vertical axis was applied on data obtained by
  an experimental \ssap\ light curve protocol using the query from
  Fig.~\ref{fig:OGLE-SC7-127550_query}. The measurements in
  Johnson-Cousins filters correspond to U (diamonds), B (asterisks),
  V (pluses) R (crosses) and I (triangles).  }
\label{fig:OGLE-SC7-127550_plot}
\end{center}
\end{figure*}

As a proper standard for displaying photometric light curves or time series
of other variables is not available from the IVOA, there have been several
attempts to exploit the visual similarity of spectra and light curves
pretending the spectral axis is the temporal one.  So the \ssap\ was
successfully used as a transport protocol for delivering light curves in
binary table format. In principle the table may contain many columns displayed
as dependent variable against the various time axes with interactive menu.
The current limitation only requires the time axis expressed as floating point
variables, which limits the possibility of using strings e.g., dates, time
stamps or even the names of CCD images used to extract given photometric point
on light curve.  An example in Fig.~\ref{fig:OGLE-SC7-127550_query} shows the
query and corresponding light curves obtained  (Fig.~\ref{fig:OGLE-SC7-127550_plot})  from
the Ond\v{r}ejov Southern Photometry Survey \citep{skoda_adassxxiii} ongoing
at the Danish 1.5m telescope DK154.

\subsection{New Features in Spectral Analysis}

\label{davids_functions}
To improve the analytic capabilities and general usability the following
features were added to the non-VO capabilities of \splatvo\ \citep{and146bcthesis}:

\subsubsection{Saving Spectra Stack in Multi-extension FITS format}

Previous versions of \splatvo\ were able to save the global list of spectra
to a binary format, which was in fact a dump of the native serialization of spectra
objects. Sometimes, there may be a need to save that list to a more
universal, standardized format. For example, \splatvo\ has many
operations on spectra, which result in a new spectrum that is added to
a global list of spectra (e.g., cutting a part of spectrum). Saving the list of
such spectra to a standardized format would allow there use in other
tools offering capabilities that \splatvo\ does not support.
This may be helpful for the preparation of large lists of normalized line profiles
for Fourier disentangling or asteroismology studies.

\splatvo\ now allows the user to save the global list of spectra to a
much more universal FITS format, where every spectrum  is represented by its
own (IMAGE) extension.  The only disadvantage compared to the native dump
format is that settings related to the way that the spectra are rendered are
no longer available.
\subsubsection{Visual Spectrum Selection and Highlighting}

In earlier versions of \splatvo, when working with multiple spectra displayed
in one common plot, it was difficult to work out which spectrum corresponded
to one in the global list, so that it could be worked on independently.
Also the selection of, for example, a noisy spectrum and its removal from
the plot window could be done only by a trial-and-error procedure and therefore
was quite problematic and time-wasting.

This has been improved and now when a spectrum is selected in the global
list, \splatvo\ will highlighted it in all the plot windows that it is displayed in.
The highlighting has the form of a few-seconds blinking in an inverted colour
and is performed sequentially (highlighting in one plot window comes after the
previous one is finished), which gives to the user a perfect sequential overview
of the spectrum's plots.

\subsubsection{Cut window: saving the ranges}

In the cut regions tool, it is possible to read the ranges of regions from a
local text file. But what if a visual selection of ranges from a currently
plotted spectrum is to be saved?  \splatvo\ can now save the currently
defined ranges to a local file in the same format as used for reading.

\subsubsection{Cut window: working with multiple spectra}

Previously any regions defined by the cut regions tool only operated on one
spectrum (known as the current spectrum). This tool now has a table of all
currently plotted spectra which can now be selected so that any regions are
applied to them all.

\subsubsection{Display of all spectra received by SAMP to the same window}

In previous versions of \splatvo, all spectra received via \samp\ messages
were opened in their own plot windows. Since it would be useful
to also display spectra in one plot to make comparisons, a new control to
toggle this behaviour on and off has been added.

\section{Lessons learned}

During the development of \splatvo\ a number of lessons have been
learned that may affect future developments of similar tools. In this
section we summarize these issues.

\subsection{Using JNI versus ``pure'' Java}
\label{sec:jniast-lesson}

The question should probably be asked if using a JNI wrapper to the
AST library was a good choice or not. As with most things, we believe
this was a good idea and also bad. The judgement of how bad depends on
how the ultimate cost benefit analysis works in the longer term, but
either way it is difficult to deny that a pure Java solution would
have been preferable.

The positives of this choice are that it made the development of
\splatvo\ possible on useful timescales as we were able to immediately
benefit from all the work that had gone into the AST library, as AST
is very much more than just a system for handling coordinates and
units. It also insulates against the often messy need to parse FITS
headers and offers the ability to transform easily between all
compatible coordinates and units. Even today some 15 years later, AST
is still believed to be a class leader in these areas.

Avoiding this work was also pragmatic as a re-write into Java would
have been a lot of work that the effort just wasn't available for and
still isn't. In the intervening years there have also been new
features added to AST, such as the dual-sideband support, which was
vital for submillimeter work, and the adoption of \splat\ by the
Joint Astronomy Centre (JAC). The development of \splat\ has also been a useful testing ground
for the development of AST itself (like the introduction of thread
safety, so not just the handling of spectral coordinates), closing a
virtuous circle.

The downsides are only the obvious ones, not having a pure Java
solution compromises the portability, so we need to make an additional
effort for all the supported platforms. There is also the non-trivial
time writing the JNI layer itself takes. Writing JNI interface code is
more challenging than those more commonly found for scripting
languages.

\subsection{Open sourcing from the beginning}

The decision was made very early on for the \splat\ source code to be
made publicly available, first via CVS on a Starlink server, then via
Subversion on a Joint Astronomy Centre server and finally using \texttt{git}
on Github. We feel that this openness contributed directly to the
software being picked up by the GAVO project, saving the application
from stagnation once the direct funding for it was dropped.

\subsection{Local spectral line catalogues are not sufficient}

Currently \splatvo\ uses a local text file containing molecular
transitions that are thought to be of interest and allows the user to
provide external lines via a text file of a similar format. The
internal list is dominated by sub-millimetre lines required by the
JCMT. Even so, the list of lines was never sufficient and astronomers
continue to ask for more obscure transitions to be included. A local
line catalogue is an excellent back up but it is also necessary to
look up catalogues from remote services. The ALMA-OT
\citep{2013ASPC..475..373W} works in exactly this way with much
success. This feature will be added in the future using the Simple
Line Access Protocol (SLAP) that will allow for retrieval of line
lists from big world-wide atomic and molecular database such as NIST
\citep{NIST_ASD,2012APS..DMP.D1004K,2004JPCRD..33..177L} and all the
databases from the VAMDC consortium \citep{2011BaltA..20..503K}.


\subsection{SPLAT-VO in data reduction pipelines}

Interactive user interfaces are very powerful but a number of issues
become apparent when reduction and analysis facilities are directly
integrated into an application where automation is a secondary
concern.  At the James Clerk Maxwell Telescope \splat\ replaced the
\specx\ package \citep[][\ascl{1310.008}]{specx} which had been
developed for interactive command-line use whilst supporting a
scripting language to allow for bulk processing of spectra. With \splat\
the promise of beanshell support was never able to overcome some key
difficulties associated with automation. There are two scenarios for
automation. The first is for an astronomer to record the steps taken
manually in reducing or analysing a spectrum and then replay that
analysis on many other spectrum. The second scenario is to allow key
algorithms to be made usable in a pipeline environment with the GUI
disabled. For the JCMT heterodyne pipeline
\citep[][\ascl{1310.001}]{2008ASPC..394..565J,JennessACSISDR} all efforts at algorithmic
enhancement were focused on individual Starlink applications such as
\KAPPA\ \citep[][\ascl{1403.022}]{sun95}, because it was unclear how
feasible it would be to add complex algorithms to \splat\ and make them
available through the pipeline. One of the interesting approaches
taken by \gaia\ \citep{2009ASPC..411..575D} was that all analysis
algorithms were called through a messaging interface such that they
could be used by the interactive GUI, from the command-line or from
the pipeline. This proved to be a very powerful approach but was not
available in the \splat\ design. Taking care to separate core algorithms
from interactive interfaces is an important design decision that has
long term ramifications for an application.

\subsection{Sub-millimeter units are complicated}

\vounits\ \citep{vounits} standardisation is very useful but is not
sufficient for the majority of spectral line data that is encountered
by sub-millimeter astronomers. Sub-millimeter spectra are generally
calibrated in temperature rather than janskys and there are two
competing standards for handling the different calibration factors
that are involved in converting from such definitions as corrected
antenna temperature, main-beam temperature and antenna temperature
\citep{1981ApJ...250..341K,1989LNP...333..351D,2009tra..book.....W}. The
units alone do not tell you which temperature scale is in use and many
efficiency values are required when comparing spectra from different
telescopes. The situation is in dire need of standardisation efforts
and we simply had to ignore the issue in \splat.

\subsection{Service selection interface}

The new \ssap\ service selection interface, using more metadata
information from the services, is meant to help the user to choose the
specific services needed and to avoid making unnecessary queries and
downloading data which is not needed. Sometimes too much detail may
lead to confusion, especially if the service information from the
registries is often incomplete. If the data is not there as it should
be a detailed interface won't do much. The current interface will be
redesigned in the future, to be probably simpler but more accurate. It has
been planned also to use \regtap\ \citep{regtap} to look for spectral services, which
may provide easier handling.

\subsection{Tracking VO standards is hard}

\splat\ transitioned from a local analysis tool to a VO client during
the birth of the VO and witnessed an explosive growth of standards
during that time. One interesting outcome of this was that funding for
\splat\ became scarce as the VO grew and funds were diverted from
classical astronomy software development efforts. VO standards became
increasingly important and it was clear that support should be added
for \ssap\ and related standards although it was a struggle to
prioritise this effort as VO adoption was initially slow and
astronomers were not driving the initial support for these
facilities. \splatvo\ is being updated to add support for standards
such as \datalink\ \citep{datalink}, \obscore\ \citep{obstap} and \regtap\ \citep{regtap} as well as expanded units support with
\vounits\ \citep{vounits} required to understand the metadata from these
services, but the VO standards will continue to evolve
as new versions are announced and new standards developed and it will
be a continuing struggle to remain compliant. There is a worry that
all of the development effort available to \splatvo\ will be spent
simply on maintaining standards support and this will be at the
expense of supporting new analysis algorithms. This is a very
difficult balancing act and we can not expect the IVOA standards
process to stagnate purely to make life easier for applications
developers.

\subsection{The Client is not Omnipotent}

During the discussions about VO standards we often hear that the
protocol should be very simple, focusing just on allowing users to
query the particular data set and express the location of original
files, with all the additional post-processing of data left as a task
for the client.

Unfortunately, when we started using \splatvo\ and a simple \ssap\ server
of stellar spectra from Ond\v{r}ejov HEROS observations, we
immediately faced the limitations of purely client-based
processing. Every spectrum of HEROS \citep{2002PAICz..90....1S} is more than twenty thousand
points long covering about 3000\,$\AA$ in both channels. Upcoming
instruments from sub-millimeter telescopes such as CCAT
\citep{jenness_adassxxiii} may have a hundred thousand channels per
spectrum and echelle spectra (like those from UVES
\citep{2000SPIE.4008..534D} or HARPS \citep{2000SPIE.4008..582P}) may have
as many as three hundred thousand points. Just downloading thousands
of spectra takes a long time, but zooming on every spectral line to
see it in detail and unzooming takes even longer, while consuming lot
of memory.

This practical experience demonstrated the analysis of a stack of hundreds
of complete spectra in \splatvo\ on a common computer, was a very inconvenient and
time consuming task (if possible at all), while usually only a short spectral
range of tens of Angstroms was needed to display each time.  If the server
itself (usually a very powerful computer with a lot of memory) would trim to
the required range, the client need download only a very small data chunk,
which could be quickly displayed.  Instead of spending time zooming to another
line we can in principle perform another query in different spectral range
requesting again only a short piece of another line range.

The need for server-side post-processing, as articulated by our
experience with analysis of thousands of spectra lead to the first
implementation\footnote{\url{http://wiki.ivoa.net/internal/IVOA/InterOpOct2008DAL/stelSSAcutout.pdf}}
of a VO-based post-processing server, based on the \pleinpot\ suite
\citep{2005ASPC..347..385C}, with simple on-the-fly cutouts and
relative rescaling of spectra, later on ported to \dachs\ and
supplemented by automatic continuum
normalization\footnote{\url{http://wiki.ivoa.net/internal/IVOA/InteropMay2012DAL/champaign-getdata.pdf}}.
As \ssap\ does not contain support for post-processing, it was decided
pragmatically to ``overload'' the \ssap\ query parameters representing
query and processing parameters at the same time.

This required to set up another (post-processing) \ssap\ server, where
the \texttt{BAND} limits are used for cutting of the given spectral
range (previously selected) and \texttt{FLUXCALIB} was understood as a
command to divide by the maximum value (\texttt{FLUXCALIB=relative})
or call automatic continuum normalization
(\texttt{FLUXCALIB=normalized}).  This allowed us to display detailed
profiles of hundreds of spectra, including the very long echelle
spectra, study line profiles evolution in long time-series or identify
emission episodes in a long series of Be stars observations just by
overplotting hundreds of on-the-fly normalized cuts of the H$_\alpha$
line.  Although this solution is not consistent with original VO
standard motivation, it served its purpose and resulted in the
\texttt{getData} proposal, finally replaced by \datalink, even though
the original \datalink\ was not conceived as post-processing mechanism
(rather it should just point to alternative version of images or
previews etc.).  In any case it is clear that the client is generally
not able to handle large amounts of spectra, due to lack of memory and
processing power. The properly balanced design separation of tasks
between client and server allows for comfortable processing of really
Big Data.  A very nice and clean protocol, which cannot handle real
scientific needs, must finally be made more complex by
extensions. Even if it is nice idea, we cannot put all the processing
burden on the client.  This is always the difficult trade-off between
design simplicity and scientific requirements.


\section{Obtaining SPLAT-VO}

\splatvo\ is released in a variety of ways, you can build it using the source
code, obtain it as part of a Starlink JAC release. Alternatively you can
install into your desktop using a standalone IzPack bundled version or run it
up using Java webstart.

The bundled standalone and webstart releases of \splatvo\ are currently
available for the following desktops, Linux (32 and 64 bit), Mac OS X (32 and
64 bit Lion) and Windows XP and later (32 bit Java only). Previous versions
also worked on Solaris, Sparc and Intel, and OS X PPC.

The Starlink releases
\citep[e.g.,][]{currie_adassxxiii,2013ASPC..475..247B} are available
for Linux (32 and 64 bit) and OS X (Lion and Snow Leopard).

See \ascl{1402.007}\footnote{or \url{http://astro.dur.ac.uk/~pdraper/splat}}
for the main support site where links to these releases can be found.
Releases of recent GAVO developments are also available and these
developments will be merged into the main \splatvo\ releases.

The source code for \splatvo\ and its associated libraries is
open-source and available on Github at
\url{https://github.com/Starlink/starjava}


\section{Conclusions}

\splatvo\ is a very powerful tool for the analysis of spectra allowing the
immediate browsing of world-wide distributed archives as well as detailed
studies of individual objects across the whole electromagnetic spectrum by the
building of SEDs from spectra taken in many different enery bands. Its
capabilities make it an ideal tool for the analysis of astronomical spectra on
the local desktop, facilities that are promoted to full productivity when
combined with resources from the VO environment. It is currently under active
development, promising soon handling of light curves and data cubes. We
also intend to add support for automatic matching of line identifiers to
spectra, simultaneous processing of multiple spectra, automated continuum
normalization, and enhanced support of FITS extensions and metdata.

As a result of combining the effort of GAVO \dachs\ server suite
developers and \splatvo\ developers, \splatvo\ is an ideal reference
implementation for testing new IVOA proposed standards, as required by
IVOA rules, for pushing standards in the recommendation phase.

\section*{Acknowledgements}

The initial work on \splatvo\ was supported
by the now closed Starlink Project funded by the Particle Physics and
Astronomy Research Council and more recently by the Joint Astronomy
Centre, Hawaii, also funded by the Particle Physics and Astronomy
Research Council and more recently by its successor organisation the
Science and Technology Facilities Council.

The current work on \splatvo\ by GAVO is supported by German Federal
Ministry of Education and Research, BMBF grant 05A11VH3 and the Czech
development is supported by grant 13-08195S of Granting Agency of the
Czech Republic and by the Czech project RVO:67985815.

We thank Mark Taylor, who added the \samp\ support to \splatvo\ and created the
VO Registry and \tap\ query libraries used (these are part of StarJava).  We
also thank Markus Demleitner who actively supports the new features in
\splatvo\ \ssap\ and \datalink\ handling by the development of GAVO \dachs\
server publishing suite.  Finally we thank Jan Wouterloot for his work testing
the sub-millimetre features of \splatvo.

Many new features of \splatvo\ are based on everyday experience with
analysis of hot emission line stars spectra observed at Perek 2m
Telescope of the Ond\v{r}ejov observatory of the Astronomical Insitute
of the Academy of Sciences, Czech Republic. We greatly acknowledge the
help of M.~\v{S}lechta, responsible for reduction of most of the
spectra and students T.~Peterka and J.~N\'advorn\'\i{}k for the set up
and maintainance of the \ssap\ services of the AI stellar department,
with which \splatvo\ was heavily tested in recent years.

\bibliographystyle{model2-names-astronomy}
\bibliography{acsplat}


%!!! explain what is OSPS and cite ADASS article !!!
%bc thesis of Andresic




%\begin{figure*}
%\begin{center}
%\includegraphics[width=0.8\textwidth]{}
%\caption{}
%\label{fig:}
%\end{center}
%\end{figure*}



\end{document}
